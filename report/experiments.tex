\section{Experiments}
\label{exp}
%introduction on how this chapter is structured

%Experimental Results (recommended size: 1.5 pages)
\subsection{Set-up}
%a. Experimental setup: describe the working environments (DAS, Amazon EC2,
%etc.), the general workload and monitoring tools and libraries, other tools and
%libraries you have used to implement and deploy your system, other tools and
%libraries used to conduct your experiments.
All of our experiment were run on Amazon EC2 t2.micro machines, these machines have only 1 CPU and 1GB of memory. We had rented 5 of them and had used them for all our experiments. However, it is very easy to extend our application to support unlimited number of Virtual Machines. We were running Amazon Linux with Java installation on all of our instances.
\subsection{Experiment analysis}
%b. Experiments: describe the experiments you have conducted to analyze each
%system feature, then analyze them; use one sub-section per experiment. For
%each experiment, describe the workload, present the operation of the system,
%and analyze the results.

 %In the analysis, report:
%i. Charged-time = time that would have been charged using the
%Amazon EC2 timing approach (1-hour increments)
%ii. Charged-cost = cost that would have been charged using the
%Amazon EC2 charging approach, assuming 10 Euro-cents/charged hour
%iii. Service metrics of the experiment, such as runtime and response time of
%the service, etc.
%iv. (optional) Usage metrics of the experiment, such as per-VM and overall
%system usage and activity.
\subsubsection{Automation}

\subsubsection{Elasticity}

without provisioning vs with, show that it isn't much slower than just having the jobs all on

\subsubsection{Load balancing}

(job max on a slave)
turn off load balancing vs on
\subsubsection{Reliability}

kill job, show that user can submit again
\subsubsection{Monitoring}

\subsubsection{MultiTenancy}
run x clients with the same job, show that variance isn't too high
