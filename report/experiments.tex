\section{Experiments}
\label{exp}
%introduction on how this chapter is structured

%Experimental Results (recommended size: 1.5 pages)
\subsection{Set-up}
%a. Experimental setup: describe the working environments (DAS, Amazon EC2,
%etc.), the general workload and monitoring tools and libraries, other tools and
%libraries you have used to implement and deploy your system, other tools and
%libraries used to conduct your experiments.
\subsection{Experiment analysis}
%b. Experiments: describe the experiments you have conducted to analyze each
%system feature, then analyze them; use one sub-section per experiment. For
%each experiment, describe the workload, present the operation of the system,
%and analyze the results.

 %In the analysis, report:
%i. Charged-time = time that would have been charged using the
%Amazon EC2 timing approach (1-hour increments)
%ii. Charged-cost = cost that would have been charged using the
%Amazon EC2 charging approach, assuming 10 Euro-cents/charged hour
%iii. Service metrics of the experiment, such as runtime and response time of
%the service, etc.
%iv. (optional) Usage metrics of the experiment, such as per-VM and overall
%system usage and activity.
\subsubsection{Automation}

\subsubsection{Elasticity}

\subsubsection{Load balancing}

\subsubsection{Reliability}

\subsubsection{Monitoring}

