\section{Experiments}
\label{exp}
%introduction on how this chapter is structured

%Experimental Results (recommended size: 1.5 pages)
\subsection{Set-up}
%a. Experimental setup: describe the working environments (DAS, Amazon EC2,
%etc.), the general workload and monitoring tools and libraries, other tools and
%libraries you have used to implement and deploy your system, other tools and
%libraries used to conduct your experiments.
All of our experiment were run on Amazon EC2 t2.micro machines, these machines have only 1 CPU and 1GB of memory. We had rented 5 of them and had used them for all our experiments. However, it is very easy to extend our application to support unlimited number of Virtual Machines. We were running the Amazon Linux operating system with a Java installation on all of our instances.

To perform the experiments, we have prepared 5 different ZIP files with different size containing different amount of images. The overview of this test set is in the table \ref{testset}.

\begin{table}
\centering
 \label{testset}
 \begin{tabular}{| l | c | c |}
  \hline
  Set name & ZIP size & Number of images \\
  \hline
  \hline
  1 & 1MB & 1\\
  2 & 5MB & 5\\
  3 & 10MB & 10\\
  4 & 50MB & 50\\
  5 & 100MB & 100\\
  \hline
 \end{tabular}

 \caption{Files used for experiments.}
\end{table}

\subsection{Experiment analysis}
%b. Experiments: describe the experiments you have conducted to analyze each
%system feature, then analyze them; use one sub-section per experiment. For
%each experiment, describe the workload, present the operation of the system,
%and analyze the results.

 %In the analysis, report:
%i. Charged-time = time that would have been charged using the
%Amazon EC2 timing approach (1-hour increments)
%ii. Charged-cost = cost that would have been charged using the
%Amazon EC2 charging approach, assuming 10 Euro-cents/charged hour
%iii. Service metrics of the experiment, such as runtime and response time of
%the service, etc.
%iv. (optional) Usage metrics of the experiment, such as per-VM and overall
%system usage and activity.

\subsubsection{Elasticity}
\label{elastic}
In this experiment we simulated a steady interval of clients connecting and sending jobs to our application, one with all slaves already turned on and one with our VM Manager activated and all slaves turned off initially. All the jobs have the image size of TODO and the clients connect at a rate of TODO.

%TODO insert graph here

\subsubsection{Load balancing}
In this experiment we have simulated two scenarios, a flash crowd arrival of a number of clients with a random file out of Table \ref{testset} and a steady state scenario where clients connect every 5 seconds. %TODO insert graph of with and without load balancing for certain number of users

\subsubsection{Reliability}
When a machine fails, the client gets notified about this and has the possibility to resend the job. 

\subsubsection{Multi-Tenancy}
As seen in Section \ref{elastic}, multiple users can connect and run their job. By provisioning more machines, more users can be serviced.
